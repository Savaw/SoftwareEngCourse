%نام و نام خانوادگی:
%شماره دانشجویی: 
\مسئله{}
نرم‌افزار 
\lr{App Store}
برای اپلیکیشن‌های موبایلی را در نظر بگیرید. یکی از تعاملاتی که کاربر با این نرم‌افزار دارد، جست‌وجوی برنامه‌ها در آن است.

\begin{enumerate}[a)]
	\item 
	نحوه تعامل کاربر با نرم‌افزار را در این مورد کاربری به صورت غیر رسمی بنویسید.
	 
	 \item 
به توصیف غیر رسمی خود که در قسمت قبل نوشتید، حداقل سه مورد
\lr{Exception}
اضافه کنید و سپس این مورد کاربری را در قالب رسمی بنویسید.
\end{enumerate}


\پاسخ{

 
 
 \begin{enumerate}[a)]
 	\item 
 	
کاربر وارد نرم‌افزار می‌شود و محلی که برای جست‌وجو در نظر گرفته شده را پیدا می‌کند. کاربر در این باکس اقدام به ارائه‌ی نام برنامه‌ی مد نظر به صورت متنی یا گفتن نام نرم‌افزار به صورت صوتی می‌نماید. کاربر مایل است در صورتی که قسمتی از نام را بنویسد، نرم‌افزار پیشنهادهایی برای تکمیل نام به وی بدهد. پس از ارائه‌ی نام یا قسمتی از نام برنامه به صورت متنی یا صوتی، کاربر انتظار دارد نرم‌افزار برنامه‌هایی را که تطابق بیشتری با عبارت جست‌وجو شده دارند را به ترتیب به وی نشان دهد و با انتخاب هر کدام از آن‌ها امکان مشاهده‌ی مشخصات و نصب آن‌ها را داشته باشد.

 	\item 
 	
 	\lr{Exceptions}:
 	
		
\begin{enumerate}[1.]
	\item
	عدم رعایت محدودیت سنی - 
		پیش از نمایش لیست نهایی به کاربر، در صورتی که برخی از برنامه‌ها محدودیت سنی خاصی داشته باشند که مشخصات کاربر آن‌ها را رعایت نکند، آن برنامه‌ها از لیست برنامه‌های بازگردانده شده حذف می‌شوند.
	\item
		عدم دسترسی به علت موقعیت مکانی -
	پیش از نمایش لیست نهایی به کاربر، در صورتی که برخی از برنامه‌ها در موقعیت مکانی کاربر در دسترس نباشند، آن برنامه‌ها از لیست برنامه‌های بازگردانده شده حذف می‌شوند.
	\item
	یافت نشدن برنامه مطابق با معیارهای جست‌وجو -
	در صورتی که پس از ورود معیارهای جست‌وجو توسط کاربر، نرم‌افزار نتواند هیچ برنامه‌ای مطابق با آن‌ها را پیدا کند، باید پیامی مبنی بر یافت نشدن برنامه بدهد.
		\item
 خطای ارتباط -
	در صورتی که برنامه هنگام جست‌وجو دچار خطای ارتباط با سرور شود لازم است پیامی مبنی بر بروز خطا و لزوم تلاش دوباره بدهد.
			\item
			خطای پردازش صوت کاربر - 
			در صورتی که کاربر ارائه‌ی عبارت جست‌وجو به صورت صوتی را انتخاب کند و نرم‌افزار نتواند صوت کاربر را به متن با معنی تبدیل کند، لازم است پیامی مبنی بر خطا در دریافت صوت را به کاربر ارائه دهد و از وی درخواست کند دوباره عبارت را بیان کند.
\end{enumerate}

توصیف مورد کاربرد در قالب رسمی در جدول \ref{fig4-1} آورده شده است. با توجه به ماهیت نرم‌افزار
\lr{App Store}،
 جست‌وجوی برنامه‌ها در آن اهمیت زیادی دارد و به همین علت نرخ استفاده از این مورد کاربرد زیاد و اولویت آن نیاز بالا در نظر گرفته شده است.
 	
\newpage
\begin{center}
\begin{table}[h!]
	\centering
	\begin{tabular}{|p{0.18\linewidth}|p{0.75\linewidth}|} 
		\hline
		\multicolumn{2}{|c|}{مورد کاربرد: 
	جست‌وجوی برنامه	
	}  \\ 
		\hline
		توضیح مختصر     &     کاربر با ارائه‌ی نام یا قسمتی از نام برنامه به جست‌وجو در نرم‌افزار می‌پردازد.             \\ 
		\hline
	هدف     &     کاربر بتواند برنامه‌ی مد نظر خود را در نرم‌افزار پیدا کند.             \\ 
		\hline
		کنشگرهای اولیه  &    کاربر                  \\ 
		\hline
		پیش‌نیازها      &      کاربر وارد برنامه شده باشد.               \\ 
		\hline
		trigger     &    مورد کاربری با اقدام کاربر برای جست‌وجوی یک برنامه در نرم‌افزار آغاز می‌شود.            \\ 
		\hline
		روند اصلی       &   
		
\begin{enumerate}[1.]
\item
کاربر می‌تواند جست‌وجوی متنی یا صوتی را انتخاب کند.
\item
\textbf{اگر}
 کاربر جست‌وجوی متنی را انتخاب کند:
	\begin{enumerate}[a.]
		\item
	کاربر در محلی که نرم‌افزار برای نوشتن به وی ارائه می‌دهد اقدام به نوشتن نام برنامه می‌کند.
		\item
	نرم‌افزار به کاربر گزینه‌های پیشنهادی برای تکمیل عبارت جست‌وجو رو پیشنهاد می‌دهد.
	\end{enumerate}
\item
\textbf{در غیر این صورت:}
	\begin{enumerate}[a.]
		\item
		کاربر به صورت صوتی نام یا قسمتی از نام برنامه را ذکر می‌کند و نرم‌افزار صدای وی را ضبط می‌کند.
	\end{enumerate}

\item
نرم‌افزار با توجه به معیارهای جست‌وجوی کاربر لیستی از برنامه‌ها را به ترتیب مرتبط بودن به کاربر نشان می‌دهد و امکان مشاهده مشخصات و سایر عملیات‌های مرتبط با برنامه را برای هر کدام از برنامه‌ها فراهم می‌کند.

\end{enumerate}
		                   \\ 
		\hline
	Exceptions   &    
	     
\begin{enumerate}[1.]
	\item
	عدم رعایت محدودیت سنی 
	\item
	عدم دسترسی به علت موقعیت مکانی
	\item
	یافت نشدن برنامه مطابق با معیارهای جست‌و‌جو
	\item
	خطای ارتباط
	\item
	خطای پردازش صوت کاربر
\end{enumerate}
\\
	\hline
نرخ استفاده	&    زیاد        \\ 
	\hline	
	اولویت	&    بالا        \\ 
	\hline	
\end{tabular}
	\caption{مورد کاربرد جست‌وجوی برنامه}	
		\label{fig4-1}
\end{table}


\end{center}

 	

\end{enumerate}
 
	


\subsection*{مراجع}

\begin{latin}
	\begingroup
	\renewcommand{\section}[2]{}%
	
	\begin{thebibliography}{9}
		%   Check this for adding items: https://www.student.unsw.edu.au/how-do-i-cite-electronic-sources
		
		\bibitem{pressman}
		R. Pressman,   B. Maxim, (2014),
		\textit{Software Engineering: A Practitioner’s Approach, 8th Ed. },
		McGraw-Hill.

	\bibitem{up}
	Arlow, J., \& Neustadt, I, (2005).,
		\textit{UML 2 and the unified process: practical object-oriented analysis and design.},
		 Pearson Education.		
		
	\end{thebibliography}
	\endgroup
\end{latin}

}
