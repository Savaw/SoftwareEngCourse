%نام و نام خانوادگی:
%شماره دانشجویی: 
\مسئله{
}
تحقیق نمایید:

\begin{enumerate}[a)]
	\item 
چه خطاهایی \lr{Server-side} هستند؟
\item 
چه خطاهایی \lr{Client-side} هستند؟
\item
کدام یک از این دو نوع خطا جدی‌تر است؟ چرا؟
\end{enumerate}


\پاسخ{
\begin{enumerate}[a)]
    \item خطاهای server-side جزو مشکلات حیاتی در توسعه وب هستند، زیرا می‌توانند بر عملکرد اصلی یک وب‌سایت یا نرم‌افزار تأثیر بگذارند. این خطاها در سرور رخ می دهند و می توانند ناشی از عوامل مختلفی از جمله مشکلات مربوط به کد سمت سرور، پیکربندی نامناسب سرور، یا مشکلات در محیط سرور و … باشند. مثلا سروری که در دسترس نیست یا پاسخ نمی‌دهد، اسکریپت سمت سرور که خطا ایجاد می‌کند، یا مشکل در پایگاه داده‌ای که سرور به آن متکی است همگی خطاهای \lr{Server-side} خواهند داد. بررسی خطاهای \lr{Server-side} بسیار مهم است تا از نشستی پایدار و قابل اعتماد برای کاربران خود اطمینان حاصل کنیم.
    \item خطاهای \lr{Client-side} نوعی مشکل در توسعه وب است که در سمت دستگاه یا مرورگر کاربر رخ می‌دهد. این نوع خطا ناشی از عوامل مختلفی می‌تواند باشد؛ مثلا ورودی نادرست کاربر در فیلد‌های یک فرم یا مشکلات کد \lr{Client-side} (مثل جاوا اسکریپت) . رسیدگی به این خطاها برای ارائه \lr{UX} راحت و بدون درز اهمیت دارد.
    \item خطاها، چه \lr{Client-side} و چه \lr{Server-side}، می‌توانند سطوح مختلفی از شدت و تاثیر بر \lr{UX} یا پایداری یک برنامه داشته باشند. به طور کلی خطاهای \lr{Server-side} به دلیل دامنه‌ی گسترده‌تر و پتانسیل لو دادن اطلاعات حساس، مانند داده های کاربر، جدی‌تر در نظر گرفته می شوند. به عنوان مثال، یک خطای \lr{Server-side} که باعث اختلال در عملکرد پایگاه داده می‌شود، می‌تواند منجر به اختلال گسترده و حتی از دست دادن داده‌ها شود. اما در خطاهای \lr{Client-side} احتمالا تنها یک کاربر تحت تاثیر قرار می‌گیرد و معمولا از نظر امنیت داده‌ها اهمیت کمتری دارند.
    \\
    با این حال، خطاهای \lr{Client-side} را نباید دست کم گرفت. این خطاها می توانند باعث ناامیدی کاربر شده و بر \lr{UX} تأثیر منفی بگذارند. مثلا وقتی دچار خطای \lr{Client-side} شویم و با ارسال یک خطای نامرتب از ارسال صحیح فرم جلوگیری کنیم، کاربر دچار تجربه‌ای ناامیدکننده خواهد شد.
\\
هر دو نوع خطا را باید مناسب مدیریت کرد، زیرا می‌توانند تأثیر قابل توجهی بر \lr{UX} و ثبات کلی برنامه داشته باشند. در سمت کلاینت، اعتبارسنجی ورودی کاربر و ارائه‌ی پیام‌های خطای معنی‌دار مهم است و در سمت سرور، نظارت و ثبت خطاهای سمت سرور مهم است تا بتوان سریعا به آنها رسیدگی و از مشکلات بعدی جلوگیری کرد.
\end{enumerate}

\subsection*{مراجع}

\begin{latin}
	\begingroup
	\renewcommand{\section}[2]{}%
	
\begin{thebibliography}{9}
%   Check this for adding items: https://www.student.unsw.edu.au/how-do-i-cite-electronic-sources	

	
	\bibitem{MAGNETIC}
	\textit{Magnetic Creative Website},
	Accessed on 2/5/2023,
	\url{https://know.magneticcreative.com/client-side-errors}

\end{thebibliography}
\endgroup
\end{latin}

}
