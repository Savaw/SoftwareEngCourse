%نام و نام خانوادگی:
%شماره دانشجویی: 
\مسئله{
}
تحقیق نمایید:

\begin{enumerate}[a)]
	\item 
چه خطاهایی \lr{Server-side} هستند؟
\item 
چه خطاهایی \lr{Client-side} هستند؟
\item
کدام یک از این دو نوع خطا جدی‌تر است؟ چرا؟
\end{enumerate}



\پاسخ{
\begin{enumerate}[a)]
    \item خطاهای \lr{Server-side} در سمت سرور رخ می‌دهند و به دلیل مشکلات مختلفی از جمله کد سمت سرور، پیکربندی سرور یا محیط سرور است. نمونه‌هایی از خطاهای \lr{Server-side} عبارتند از: سروری که از کار افتاده یا پاسخگو نیست، اسکریپت سمت سرور که خطا ایجاد می‌کند، یا مشکلی در پایگاه داده سرور.
    \item خطاهای \lr{Client-side} در سمت کامپیوتر کاربر رخ می دهند و معمولاً به دلیل ورودی نادرست یا سایر مشکلات کد سمت مشتری، مثل \lr{Javascript}، است. نمونه‌هایی از خطاهای \lr{Client-side} شامل وارد کردن داده‌های نامعتبر توسط کاربر در فرم یا مشکل در اسکریپت سمت کاربر است که مانع از عملکرد صحیح صفحه می‌شود است.
    \item هر دو خطای سمت کلاینت و سمت سرور بسته به زمینه و تأثیری که بر \lr{UX} و پایداری کلی برنامه دارند، می توانند جدی باشند. اما به طور کلی خطاهای سمت سرور معمولاً جدی‌تر در نظر گرفته می‌شوند زیرا قابلیت تاثیرگذاری بر تعداد بیشتری از کاربران را دارند و همچنین اطلاعات حساسی مانند داده‌های کاربر را به خطر می‌اندازند. مثلا اگر یک خطای سمت سرور منجر به از دسترس خارج شدن پایگاه داده شود، این می‌تواند باعث اختلال گسترده در برنامه و منجر به از دست رفتن داده‌ها شود.
\\
از سوی دیگر، خطاهای سمت کلاینت ممکن است تنها یک کاربر را تحت تأثیر قرار دهند و اغلب از نظر امنیت داده‌ها اهمیت کمتری دارند. با این حال، این نوع خطاها هم می‌توانند تأثیر قابل توجهی بر تجربه کاربر داشته باشند و منجر به ناامیدی یا سردرگمی کاربر شوند. مثلا اگر یک خطای سمت کاربر منجر به عدم ارسال صحیح فرم شود، این امر می‌تواند مانع از تکمیل یک کار توسط کاربر شود و بر تجربه‌ی کلی او از برنامه تأثیر منفی خواهد گذاشت.
\\
هر دو نوع خطا را باید به شکل مؤثر مدیریت کرد، زیرا می‌توانند تأثیر قابل توجهی بر \lr{UX} و ثبات کلی برنامه داشته باشند. در سمت کلاینت، اعتبارسنجی ورودی کاربر و ارائه‌ی پیام‌های خطای معنی‌دار مهم است و در سمت سرور، نظارت و ثبت خطاهای سمت سرور مهم است تا بتوان سریعا به آنها رسیدگی و از مشکلات بعدی جلوگیری کرد.
\end{enumerate}

\subsection*{مراجع}

\begin{latin}
	\begingroup
	\renewcommand{\section}[2]{}%
	
\begin{thebibliography}{9}
%   Check this for adding items: https://www.student.unsw.edu.au/how-do-i-cite-electronic-sources	
	\bibitem{MAGNETIC}
	\textit{Magnetic Creative Website},
        Accessed on 2/5/2023,
        \url{https://know.magneticcreative.com/server-side-errors}
	
	\bibitem{MAGNETIC}
	\textit{Magnetic Creative Website},
	Accessed on 2/5/2023,
	\url{https://know.magneticcreative.com/client-side-errors}

\end{thebibliography}
\endgroup
\end{latin}

}
