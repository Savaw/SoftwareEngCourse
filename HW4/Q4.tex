%نام و نام خانوادگی:
%شماره دانشجویی: 
\مسئله{}

تحقیق نمایید منظور از \lr{Test Automation} در آزمون وبسایت‌ها یا موبایل اپلیکیشن‌ها چیست؟ چه فوایدی دارد؟ حداقل یک نمونه ابزار برای هر یک از آن‌ها بررسی و معرفی نمایید.

\پاسخ{

به طور کلی \lr{Test Automation} به معنی خودکارسازی بخش‌هایی از فرآیند آزمون نرم‌افزار است به طوری که بدون نیاز به دخالت انسانی، فرآیند تست انجام بشود. به بیان بهتر، این فرآیند به معنی استفاده از یک نرم‌افزار مجزا برای کنترل و اجرای تست‌ها و مقایسه نتایج آن‌ها به طور خودکار است. این نوع تست‌ها هم در لایه بک‌اند و \lr{API} معنادار هستند و هم در لایه فرانت‌اند و رابط کاربری نرم‌افزار.

در مورد تست وبسایت‌ها و موبایل اپلیکیشن‌ها، بیش‌تر با قسمت‌های مربوط به رابط کاربری و همچنین موارد جانبی مرتبط با آن نظیر پرفرمنس یا امنیت سر و کار داریم. در تست وب‌اپلیکیشن‌ها مواردی نظیر تست محتوا، تست رابط و همچنین تست طراحی مطرح بودند. تست محتوا به معنی بررسی ساختار کلی صفحه است. تست رابط به معنی این است که همه \lr{Usecase} هایی که طراحی شده‌اند تست بشوند که از طریق رابط کاربری برنامه قابل استفاده باشند و تست طراحی به معنی تست برنامه برای یافتن خطاهای مربوط به ناوبری در آن است.

همه این موارد در مورد تست موبایل اپلیکیشن‌ها هم صادق هستند. با این تفاوت که تست‌های دیگری هم برای آن مطرح می‌شوند. مثلا در تست تطابق و \lr{Compatibility} و همچنین \lr{Integration Testing} به دلیل وجود تعداد زیادی موبایل مختلف، باید به شکل جامع‌تری تست‌ها انجام بشوند. همچنین تست‌های پرفرمنس باید مواردی نظیر فضای ذخیره‌سازی محدودتر، مشکلات مربوط به انرژی و مصرف توان و همچنین شبکه‌های ارتباطی در دسترس هم مد نظر قرار بگیرد. از نظر امنیتی هم موارد جدیدی در دنیای موبایل مطرح می‌شوند. علاوه بر این‌ها مواردی نظیر \lr{Accessibility} هم از جمله مواردی هستند که کمتر به آن‌ها پرداخته شده است ولی در اپلیکیشن‌های امروزی باید مورد تست قرار بگیرند. 

حال در مورد تست اتوماتیک و خودکار، این مسئله مطرح می‌شود که این موارد به صورت خودکار و بدون نیاز به دخالت انسانی و به کمک یک نرم‌افزار که خود فرایند تست را انجام بدهد انجام بشوند. این موضوع در مورد موبایل اپلیکیشن‌ها به خصوص با توجه به وجود تعداد زیاد موبایل‌هایی که وجود دارد حائز اهمیت بسیاری است. امکان تست رابط کاربری نرم‌افزارهای موبایل برای همه موبایل‌ها وجود ندارد ولی اگر ابزاری داشته باشیم که به طور اتوماتیک روی موبایل‌های فیزیکی یا شبیه‌سازی‌ شده مختلف به صورت خودکار تست‌ها را اجرا بکند، فرآیند بسیار سریع‌تر صورت خواهد پذیرفت. این مسئله در مورد وب اپلیکیشن‌ها هم به دلیل تنوع مرورگرها و همچنین سخت‌افزارهای مورد استفاده چه از نظر قدرت سخت‌افزاری و چه از نظر فرم فاکتور وجود دارد.

از جمله فواید تست‌های اتوماتیک می‌توان به موارد زیر اشاره کرد:

\begin{enumerate}
	\item 
	سرعت بالاتر در انجام تست‌ها
	
	\item 
	یافتن سریع‌تر خطاها و امکان صرف وقت صرفه‌جویی شده برای رفع مشکلات بیش‌تر
	
	\item 
	افزایش دقت تست‌ها به دلیل جلوگیری از خطای انسانی، علی الخصوص در مواردی نظیر تست \lr{GUI} که امکان وجود خطای انسانی به دلیل ماهیت کار وجود دارد.
	
	\item 
	کاهش هزینه‌ها به دلیل عدم نیاز به صرف نیروی انسانی پرهزینه برای انجام تست‌های تکراری و امکان صرف این هزینه برای طراحی تست‌های بهتر و جامع‌تر
	
	\item 
	افزایش سطح اتکاپذیری و اعتماد به برنامه به دلیل انجام تست‌ها
	
	\item 
	اطمینان حاصل کردن از کارکرد درست برنامه روی سخت‌افزار‌ها و تنظیمات مختلف (مرورگر، سیستم‌عامل و...)
	
	\item 
	افزایش روحیه افراد تیم به دلیل عدم نیاز به انجام تست‌ها به صورت دستی و تکراری در طی یک فرآیند مکانیکی خسته‌کننده
	
\end{enumerate}

در مورد تست موبایل اپلیکیشن‌ها خوب است که موارد زیر در نظر گرفته بشوند:

\begin{enumerate}
	
	\item
	نحوه تشخیص اشیا - به این معنی که اجزای مختلف صفحه به طرف مختلف نظیر کد \lr{HTML}، تشخیص متن، تشخیص و پردازش تصویر و... شناسایی بشوند.
	
	\item 
	امنیت - به این معنی که ابزار نیاز به 	اتصال یک دیوایس محافظت‌نشده به اینترنت نداشته باشد.
	
	\item 
	ابزار بتواند تا حد امکان مستقیما با دیوایس‌ها بدون نیاز به فعال کردن حالت‌های توسعه دهنده ارتباط برقرار کند تا همان تجربه نهایی کاربر را تست کند.
	
	\item 
	از همه قابلیت‌های دیوایس پشتیبانی کند. مواردی نظیر \lr{Multitouch}، کیبوردهای مجازی، دریافت اس‌ام‌اس و تماس، پردازش \lr{Alert} و... باید پوشش داده شده باشد.
	
	\item 
	امکان اجرای تست یکسان روی دیوایس‌های مختلف را به راحتی فراهم کند.
	
	\item
	در موارد لزوم امکان تست شرایطی نظیر اتصال چند دیوایس از طریق \lr{USB} یا وایرلس یا بلوتوث را برای بررسی پایداری اتصال‌ها فراهم کند.
\end{enumerate}



\textbf{معرفی ابزارها:}

موبایل:



دو تا از معروف‌ترین ابزار‌های خودکارسازی تست برای برنامه‌های موبایل، \lr{Espresso} و \lr{Appium} هستند. \lr{Espresso} ابزاری است که توسط خود گوگل توسعه یافته و بخشی از \lr{SDK} اندروید محسوب می‌شود. با توجه به این موضوع، این ابزار جزو پایدارترین ابزارهای موجود برای تست خودکار اپلیکیشن‌های اندرویدی بوده و سرعت بالاتری هم دارد. همچنین این ابزار امکان کامپایل شدن مجموعه تست‌ها در قالب یک \lr{APK} جدا که موازی با برنامه اصلی اجرا بشوند را هم می‌دهد. \lr{API} آن بسیار ساده است و کار با آن از طریق سه کامپوننت 
\texttt{viewMatcher}، \texttt{viewActions} 
و 
\texttt{viewAssertation}
 صورت می‌گیرد. همچنین به راحتی با محیط \lr{Android Studio} هماهنگ می‌شود. از معایب آن می‌توان به این اشاره کرد که تنها برای زبان \lr{Java} مناسب است و برای سایر زبان‌هایی که اپلیکیشن‌های موبایل برای آن توسعه می‌یابند مناسب نیست. 
 
 \begin{figure}[H]
 	\centering
 	\includegraphics[width=0.3\linewidth]{figs/espresso.png}
 	\caption{Espresso}
 	\label{fig:espresso}
 \end{figure}


از سوی دیگر، \lr{Appium} ابزاری است امکان ایجاد تست برای برنامه‌های وب موبایل، هیبرید و \lr{Native} را فراهم می‌کند. ساختار \lr{Appium} به صورت کلاینت‌سروری بوده و سرور آن مبتنی بر 
\lr{Selenium WebDriver}
است و کلاینت‌ آن با زبان‌های مختلفی قابل استفاده است زیرا پروتکل ارتباطی آن پروتکل
 \lr{JSON Wire}
 است. از این ابزار می‌توان هم برای تست اپلیکیشن‌های اندرویدی و هم \lr{iOS} استفاده کرد.  همچنین ساختار آن بسیار شبیه \lr{Selenium} است که در حوزه وب کاربرد دارد و افرادی که در آن حوزه با \lr{Selenium} کار کرده باشند، با این ابزار به راحتی ارتباط می‌گیرند. از معایب اصلی آن می‌توان به سرعت پایین‌تر و همچنین نصب و تنظیم پیچیده‌تر نسبت به \lr{Espresso} اشاره کرد.
 
  \begin{figure}[H]
 	\centering
 	\includegraphics[width=0.3\linewidth]{figs/appium.png}
 	\caption{Appium}
 	\label{fig:appium}
 \end{figure}
 
 وب:
 
 در حورزه وب محبوب‌ترین ابزار خودکارسازی تستی که وجود دارد \lr{Selenium WebDriver} است. این ابزار عملا اجازه خودکارسازی رفتار مرورگر‌ها و اجرای تست‌های مختلف مربتط با رابط کاربری به صورت خودکار را روی آن‌ها می‌دهد. ابزارهای بسیاری برای ارتباط برقرار کردن با آن به زبان‌های مختلف طراحی شده‌اند. این \lr{API} ها امکان دادن فرامین مختلف نظیر حرکت در صفحه، پر کردن فیلد‌های مختلف، کلیک روی قسمت‌های گوناگون و... را به \lr{Selenium} فراهم می‌کنند و آن‌ها توسط \lr{Selenium} اجرا می‌شود. اصلی‌ترین مزیت \lr{Selenium} این است که مدت بسیار زیادی است که در این حوزه حضور داشته و در نتیجه آن مستندات بسیار زیادی برای کار با آن وجود دارند و به علاوه در طول زمان به عنوان ابزار استاندارد این حوزه خود را ثابت کرده است. همچنین \lr{Selenium} تقریبا با تمامی سیستم‌عامل‌ها و مرورگرهای مورد استفاده روزمره هماهنگی دارد.
 
  معماری \lr{Selenium} به صورت چهار لایه است. در لایه اول کلاینتی که با \lr{Selenium} ارتباط برقرار می‌کند قرار دارد. در لایه بعدی پروتکل \lr{JSON Wire} قرار می‌گیرد. لایه بعدی ارتباط با درایورهای مخصوص هر مرورگر است و لایه چهارم خود مرورگر است که فرامین کنترلی مختلف از طریق درایور‌ها به آن داده می‌شوند.
  
   \begin{figure}[H]
  	\centering
  	\includegraphics[width=0.3\linewidth]{figs/selenium.png}
  	\caption{Selenium}
  	\label{fig:selenium}
  \end{figure}
 
 
 نکته جانبی:
 جالب است بدانید حتی ابزارها و مقالاتی پیرامون خودکارسازی تست‌های مربوط به \lr{Accessibility} هم وجود دارند. مثلا ابزار \lr{Latte} که نام مقاله آن در قسمت مراجع آورده شده است.
 
 

\subsection*{مراجع}

\begin{latin}
	\begingroup
	\renewcommand{\section}[2]{}%
	
\begin{thebibliography}{9}
%   Check this for adding items: https://www.student.unsw.edu.au/how-do-i-cite-electronic-sources	

	
	



	\bibitem{pressman}
	R. Pressman,   B. Maxim, (2014),
	\textit{Software Engineering: A Practitioner’s Approach, 8th Ed. },
	McGraw-Hill.

	
	\bibitem{Latte}
	N. Salehnamadi, A. Alshayban, J. Lin, I. Ahmed, S. Branham, S. Malek, (2021),
	\textit{Latte: Use-Case Assistive-Service Driven Automated Accessibility Testing Framework for Android.},
	ACM Conference on Human Factors in Computing System (CHI 2021)
	
	
	\bibitem{Android}
	\textit{Android Developer Website},
	Accessed on 2/5/2023,
	\url{https://developer.android.com/training/testing/espresso}
	
	\bibitem{Smartbear}
	\textit{Smartbear Website},
	Accessed on 2/5/2023,
	\url{https://smartbear.com/blog/appium-vs-espresso-which-framework-to-use/}
	
	\bibitem{BrowserStack}
	\textit{BrowserStack Website},
	Accessed on 2/5/2023,
	\url{https://www.browserstack.com/guide/selenium-webdriver-tutorial}

\end{thebibliography}
\endgroup
\end{latin}

}
