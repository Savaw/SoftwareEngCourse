%نام و نام خانوادگی:
%شماره دانشجویی: 
\مسئله{منظور از باگ در یک وب سایت یا موبایل اپلیکیشن چیست؟ با استفاده از مثال از سایت‌ها یا اپلیکیشن‌های مختلف 5 نمونه باگ بنویسید.\\}
 


\پاسخ{

در تست نرم‌افزار اصطلاحات متنوعی برای خطا استفاده می‌شود که در زبان فارسی همه به خطا یا باگ ترجمه می‌شوند. هنوز بر سر تعریف دقیق این اصطلاحات اختلاف نظر است و ممکن است موقع استفاده از آن‌ها شنونده آنچه که منظور شماست را متوجه نشود. این اصطلاحات را در ادامه به اختصار توضیح می‌دهیم. \cite{360logica}


\begin{itemize}
	\item خطا (Defect) :
	تفاوت بین نتایج مورد انتظار و نتایج واقعی در هنگام تست است که در واقع انحراف از نیاز مشتری است.
	
	\item خطا (Error) :
	خطا یک اشتباه، تصور اشتباه یا سوء تفاهم از سوی یک توسعه دهنده نرم افزار است. برای مثال یک برنامه نویس ممکن است نام متغیر را به اشتباه تایپ کند.
	
	\item باگ (Bug) :
	یک باگ نتیجه اشتباه در کد است. خطایی که قبل از ارسال محصول به مشتری، در محیط توسعه یافت می شود. یک خطای برنامه نویسی که باعث می شود برنامه ضعیف کار کند، نتایج نادرست ایجاد کند یا کرش کند. یک خطا در نرم افزار یا سخت افزار که باعث اختلال در عملکرد یک برنامه می شود.
	
	\item شکست (Failure) :
	ناتوانی یک سیستم یا کامپوننت نرم افزاری در انجام موارد نیاز در چارچوب نیاز‌های عملکردی و کیفی. هنگامی که یک نقص (defect) به مشتری نهایی می رسد به آن شکست (failure) می گویند. در طول توسعه، شکست‌ها معمولاً توسط تسترها مشاهده می شوند.
	
	\item خطا (Fault) :
	نتیجه خطا (error) است. 
	
\end{itemize}

\subsection*{مراجع}

\begin{latin}
	\begingroup
	\renewcommand{\section}[2]{}%
	
\begin{thebibliography}{9}
	\bibitem{360logica}
	\textit{\href{https://www.360logica.com/blog/difference-between-defect-error-bug-failure-and-fault/
	}{360logica: difference between defect, error, bug, failure, and fault}}

\end{thebibliography}
\endgroup
\end{latin}

}
