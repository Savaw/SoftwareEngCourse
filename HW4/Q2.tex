%نام و نام خانوادگی:
%شماره دانشجویی: 
\مسئله{}
فرض کنید شما در بخش آزمون نرم‌افزار (وب‌سایت یا موبایل اپلیکیشن) تیم هستید و تعدادی قابلیت (Feature) جدید توسعه یافته به شما تحویل می‌شود، فرآیندی که برای آزمون طی می‌کنید چیست؟ چه نوع تست‌هایی را می‌توانید از طریق UI محصول انجام دهید تا باگ‌های نرم‌افزار را شناسایی کنید؟

\پاسخ{
	
فرایند تست قابلیت‌های جدید را می‌توانیم با در نظر گرفتن نکات زیر انجام دهیم:

\begin{itemize}
\item
در گام نخست لازم است مطمئن شویم که قابلیت‌های جدید، کارکرد سابق برنامه و فیچرهای قبلی را دچار مشکل نکرده باشند. به این منظور 
\lr{regression testing}
انجام می‌دهیم. اگر هر کدام از قابلیت‌های قبلی برنامه دچار مشکل شده باشند لازم است نرم‌افزار به تیم توسعه بازگردانده شود تا مشکلات رفع شوند.

\item
پس از اطمینان از درستی قابلیت‌های پیشین، به سراغ تست قابلیت‌های جدید می‌رویم.
ابتدا لازم است که قابلیت‌های جدید را به طور کامل بشناسیم و بدانیم هدف هر کدام برطرف کردن چه نیازمندی‌هایی است تا بتوانیم تست‌های مناسبی طراحی کنیم.

\item 
 از آن جایی که توسعه‌دهنده‌ یا توسعه‌دهندگان یک قابلیت، قسمت‌هایی از برنامه که احتمال وجود مشکل در آن‌ها بالاتر است را می‌شناسند خوب است که با آن‌ها درباره‌ی قابلیت توسعه‌داده‌شده صحبت کرده و قسمت‌های آسیب‌پذیرتر برنامه را شناسایی کنیم.
 
 \item
 با در نظر گرفتن موارد بالا لازم است سناریوهای تست برای آزمون قابلیت‌های جدید طراحی و پیاده‌سازی شوند تا نرم‌افزار صحت‌سنجی شود.
 تست‌ها باید به گونه‌ای باشند که هم 
 \lr{validation}
 و هم
 \lr{verification}
 برنامه به خوبی صورت بگیرد. هم‌چنین تست‌ها باید پوشش بالایی داشته باشند تا باگ‌های برنامه تا حد امکان شناسایی شوند.
 
 \item
 روند تست نرم‌افزار باید از تست‌های ریزدانه به درشت‌دانه انجام شود. ریزدانه‌ترین نوع تست بستگی به این دارد که کدام تست‌ها را تیم توسعه انجام می‌دهد.
به طور معمول
\lr{unit test}
ها و
\lr{integration test}
ها توسط توسعه‌دهنده انجام می‌شود و به همین منظور تست‌های اصلی که تیم آزمون باید انجام دهد عموما شامل تست‌های
\lr{validation}
و
\lr{system}
می‌باشد. 

\item
پس از اطمینان از صحت نرم‌‌افزار و تست‌هایی که مربوط به نیازمندهای وظیفه‌ای سیستم می‌شوند، می‌توانیم سراغ تست‌های مربوط به نیازمندی‌های غیروظیفه‌ای و کیفیت نرم‌افزار برویم. از جمله‌ای این تست‌ها، تست‌های مربوط به امنیت سیستم، تست‌های کارایی، 
\lr{stress testing}،
تست‌های تجربه‌ی کاربری
و موارد این چنینی هستند که هر کدام بسته به نوع نرم‌افزار توسعه‌داده‌شده ممکن است دارای اهمیت خیلی زیاد یا اهمیت متوسط باشند.


\item
یک نوع دیگر تست نیز تست آلفا/بتا است که در صورت نیاز می‌تواند انجام شود. به طور مثال اگر از میزان استقبال از ویژگی جدید اطمینان نداریم یا این که دو گزینه برای قسمتی از قابلیت جدید داریم می‌توانیم با انتشار برنامه فقط برای بخشی از کاربران این موارد را تست کنیم.
 
  
\end{itemize}
	
	
سه نوع تست را می‌توانیم از طریق رابط کاربری محصول انجام دهیم:
\begin{itemize}
\item \lr{Exploratory Testing}:
این آزمون به صورت دستی انجام می‌شود. یک یا بیشتر آزمونگر ویژگی‌های مختلف سامانه را بررسی می‌کنند تا مطمئن شوند که درست کار می‌کنند. تمرکز این تست‌ها روی ابعاد مختلف تجربه‌ی کاربری و مسیری‌است که کاربران طی می‌کنند. به همین علت پارامترهایشان از پروژه‌ای به پروژه‌ی دیگر تفاوت می‌کند و به طبیعت سامانه‌ بستگی دارد.
\item \lr{Scripted Testing}:
این آزمون از طریق فریم‌ورک‌های اتوماسیون اجرا می‌شوند. این تست‌ها برخلاف مدل قبلی، به برنامه‌ریزی گسترده نیاز دارند و برای آن‌ها باید
\lr{test case}
هایی نوشت که پارامترها و خروجی‌های موردانتظارشان مشخص باشد تا درنهایت اسکریپت‌هایی متناسب نوشته‌شوند.
\item \lr{User Experience Testing}:
در این آزمون سامانه از دید یک کاربر تست می‌شود. این کار می‌تواند مستقیما توسط یک کاربر احتمالی انجام شود که بخشی از سامانه دراختیار او قرار داده‌می‌شود. همچنین آزمونگرها می‌توانند خودشان با کاربران صحبت کنند تا متوجه خواسته‌های آن‌ها بشوند و سناریوهای تستی برای آن‌ها بنویسند و تجربه‌ی کاربری سامانه را به این شکل بیازمایند.
\end{itemize}

\subsection*{مراجع}

\begin{latin}
	\begingroup
	\renewcommand{\section}[2]{}%
	
\begin{thebibliography}{9}
%   Check this for adding items: https://www.student.unsw.edu.au/how-do-i-cite-electronic-sources	
	\bibitem{Budgen, 1994}
	\textit{Budgen, D. “Software Design”, 1994, pp65-7}
	
	\bibitem{Toronto slides}
	\textit{Software Design Quality},
	Accessed on 1/11/2023,
	\url{http://www.cs.toronto.edu/~sme/CSC444F/slides/L12-SoftwareQuality.pdf}

	\bibitem{f-test}
\textit{What Is Feature Testing And Why Is It Important},
Accessed on 2/5/2023,
\url{	https://www.softwaretestinghelp.com/feature-testing-tutorial/}
	
	\bibitem{pressman}
R. Pressman,   B. Maxim (2014).
S\textit{oftware Engineering: A Practitioner’s Approach, 8th Ed. }.
McGraw-Hill.

\end{thebibliography}
\endgroup
\end{latin}

}
