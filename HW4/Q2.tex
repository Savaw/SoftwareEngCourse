%نام و نام خانوادگی:
%شماره دانشجویی: 
\مسئله{}
فرض کنید شما در بخش آزمون نرم‌افزار (وب‌سایت یا موبایل اپلیکیشن) تیم هستید و تعدادی قابلیت (Feature) جدید توسعه یافته به شما تحویل می‌شود، فرآیندی که برای آزمون طی می‌کنید چیست؟ چه نوع تست‌هایی را می‌توانید از طریق UI محصول انجام دهید تا باگ‌های نرم‌افزار را شناسایی کنید؟

\پاسخ{
سه نوع تست را می‌توانیم از طریق رابط کاربری محصول انجام دهیم:
\begin{itemize}
\item \lr{Exploratory Testing}:
این آزمون به صورت دستی انجام می‌شود. یک یا بیشتر آزمونگر ویژگی‌های مختلف سامانه را بررسی می‌کنند تا مطمئن شوند که درست کار می‌کنند. تمرکز این تست‌ها روی ابعاد مختلف تجربه‌ی کاربری و مسیری‌است که کاربران طی می‌کنند. به همین علت پارامترهایشان از پروژه‌ای به پروژه‌ی دیگر تفاوت می‌کند و به طبیعت سامانه‌ بستگی دارد.
\item \lr{Scripted Testing}:
این آزمون از طریق فریم‌ورک‌های اتوماسیون اجرا می‌شوند. این تست‌ها برخلاف مدل قبلی، به برنامه‌ریزی گسترده نیاز دارند و برای آن‌ها باید
\lr{test case}
هایی نوشت که پارامترها و خروجی‌های موردانتظارشان مشخص باشد تا درنهایت اسکریپت‌هایی متناسب نوشته‌شوند.
\item \lr{User Experience Testing}:
در این آزمون سامانه از دید یک کاربر تست می‌شود. این کار می‌تواند مستقیما توسط یک کاربر احتمالی انجام شود که بخشی از سامانه دراختیار او قرار داده‌می‌شود. همچنین آزمونگرها می‌توانند خودشان با کاربران صحبت کنند تا متوجه خواسته‌های آن‌ها بشوند و سناریوهای تستی برای آن‌ها بنویسند و تجربه‌ی کاربری سامانه را به این شکل بیازمایند.
\end{itemize}

\subsection*{مراجع}

\begin{latin}
	\begingroup
	\renewcommand{\section}[2]{}%
	
\begin{thebibliography}{9}
%   Check this for adding items: https://www.student.unsw.edu.au/how-do-i-cite-electronic-sources	
	\bibitem{Budgen, 1994}
	\textit{Budgen, D. “Software Design”, 1994, pp65-7}
	
	\bibitem{Toronto slides}
	\textit{Software Design Quality},
	Accessed on 1/11/2023,
	\url{http://www.cs.toronto.edu/~sme/CSC444F/slides/L12-SoftwareQuality.pdf}

\end{thebibliography}
\endgroup
\end{latin}

}
