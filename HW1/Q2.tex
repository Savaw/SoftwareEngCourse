%نام و نام خانوادگی:
%شماره دانشجویی: 
\مسئله{
}
با توجه به مدل‌های فرآیند توسعه نرم‌افزار آبشاری، نمونه‌سازی، حلزونی، افزایشی و چابک: \\
الف) شیوه رفتار هر یک از این مدل‌ها نسبت به تغییر در نیازمندی‌ها چگونه است؟ \\
ب) مزایا و معایب هر یک از مدل‌ها را خلاصه و مفید توضیح دهید.

\پاسخ{

الف) هر یک از مدل‌های داده‌شده با توجه به نحوه تعریف برخورد متفاوتی نسبت به تغییر نیازمندی‌ها خواهند داشت. به طور کلی به جز مدل آبشاری، سایر مدل‌ها یعنی نمونه‌سازی، حلزونی، افزایشی و چابک امکان تغییر نیازمندی‌ها را به طراح می‌دهند.

در مدل آبشاری امکان ایجاد تغییر در نیازمندی‌ها وجود ندارد، چرا که مرحله تعیین نیازمندی‌ها از مراحل اول این مدل است و بازگشتی به آن در فازهای بعدی وجود ندارد. بر خلاف مدل آبشاری، مدل نمونه‌سازی، تغییر در نیازمندی‌ها را ممکن می‌کند. این مدل با در نظر گرفتن ارتباط مداوم بین طراح و کاربر ارائه شده است و از آنجایی که نیازمندی‌های کاربر ممکن است از ابتدا کاملا مشخص نباشد، امکان تغییر نیازمندی‌ها برای آن در نظر گرفته شده است. مدل حلزونی نیز از این نظر مشابه مدل نمونه‌سازی است. از آنجایی که در مدل افزایشی فرایندهای تعیین نیازمندی، طراحی، پیاده‌سازی و تست برای هر ماژول به طور جداگانه طی می‌شوند، انعطاف‌پذیری بالایی در این مدل نسبت به تغییر در نیازمندی‌ها به وجود می‌آید. در نهایت مدل چابک نیز نوعی مدل افزایشی است بنابراین قابلیت تغییر خواهد داشت. 

ب) مزایا و معایب هر یک از مدل‌ها به صورت زیر است:
\begin{itemize}
\item مدل آبشاری:

مزایا:
\begin{itemize}
\item
 استفاده از این مدل ساده است.
 
\item
  فرآیند طراحی نرم‌افزار به مراحلی تقسیم شده است که نقش هر مرحله مشخص بوده و از سایر فازها متمایز است.
 
\item
   هر مرحله همراه با نتایج آن مستندسازی می‌شود. 

\end{itemize}

معایب:
\begin{itemize}
\item
 محصول تنها پس از مرحله نهایی در دسترس است.
 
\item 
بخش بازخود (\lr{Feedback}) وجود ندارد بنابراین امکان اصلاح مشکلات یک فاز در مراحل بعدی وجود نخواهد داشت.
\item
 تمام نیازمندی‌ها باید کاملا مشخص باشند زیرا امکان تغییر وجود ندارد.
\item
 این مدل برای پروژه‌های طولانی و پیچیده مناسب نیست.
\end{itemize}

\item مدل نمونه‌سازی:


مزایا:
\begin{itemize}
\item
تا زمان تولید محصول، کاربر (Client) در جریان طراحی و پیاده‌سازی قرار دارد.
\item 
در صورت تغییر یا کمبود در نیازمندی‌ها، امکان اضافه‌سازی و اعمال تغییر وجود دارد.
\item 
توابع سخت و پیچیده می‌توانند در مراحل اولیه طراحی شناسایی شوند.
\item 
از آنجایی که در این مدل یک نمونه کوچک با کاربری مناسب تولید می‌شود، کاربر درک بهتری از نتیجه نهایی خواهد داشت. 
\end{itemize}
معایب:
\begin{itemize}
\item 
از آنجایی که در هر لحظه امکان تغییر نیازمندی‌ها وجود دارد و مدل بسیار انعطاف‌پذیر است، ممکن است نتیجه نهایی بسیار پیچیده‌تر و در معیار گسترده‌تری نسبت به طراحی اولیه باشد.
\item
 هزینه صرف شده برای ساخت نمونه در نهایت تلف‌شده محسوب می‌شود چرا که از نمونه‌ها استفاده‌ای نمی‌شود و کنار گذاشته می‌شوند.
\item
  مستندسازی در این مدل به دلیل تغییر زیاد در نیازمندی‌ها ضعیف خواهد بود. 
\end{itemize}


\item مدل حلزونی:

مزایا:
\begin{itemize}
\item 
 تمرکز این مدل بر روی کم کردن میزان ریسک پروژه است بنابراین بخش‌های دارای ریسک بیشتر می‌توانند به طور موقت متوقف شده و به فازهای بعدی منتقل شوند.
\item
  زمانی که نیازمندی‌های پروژه و پیچیدگی آن نامشخص یا زیاد باشد می‌توان از این مدل استفاده نمود.
\item
  نرم‌افزار در همان مراحل اولیه از چرخه زندگی نرم‌افزار \lr{(Software Life Cycle)} ساخته‌ می‌شود.
\end{itemize}

معایب:
\begin{itemize}
\item 
مدیریت و حفظ این مدل به دلیل ساختارش پیچیده است.
\item
برای پروژه‌های کم‌ریسک و کوچک مناسب نیست.
\item
مستندسازی در این مدل نسبتا سنگین است زیرا در مراحل مختلفی صورت می‌گیرد.
\item
 موفقیت پروژه وابستگی بالایی به بررسی درست ریسک \lr{(Risk Analysis)} دارد که نیازمند تخصص است.
\end{itemize}

\item مدل افزایشی:

مزایا:

\begin{itemize}
\item 
از آنجایی که نرم‌افزار به ماژول‌های مختلفی تقسیم‌بندی می‌شود که هر کدام مدل آبشاری خود را دارند، عملیات تست و دیباگ برای هر بخش آسان‌تر می‌شود.
\item
 در هر چرخه از کار می‌توان به یک نتیجه مشهود رسید.
\item
 تغییر و افزایش نیازمندی‌ها بدون هزینه است.
\item
 زمان اولیه مورد نیاز برای تولید محصول کاهش می‌یابد. 
\end{itemize}

معایب:
\begin{itemize}
\item 

برای پروژه‌های کوچک مناسب نیست.
\item
به برنامه‌ریزی و طراحی مناسب و دقیق نیاز دارد.
\item
هزینه نهایی از هزینه انجام کل روند به صورت آبشاری گران‌تر خواهد بود.
\item
نتیجه نهایی مشخص نیست و احتمال خطا دارد (ریسک بالا).
\end{itemize}
\item مدل چابک:

مزایا:
\begin{itemize}
\item 

برای هر دو نوع کار انعطاف‌پذیر و ثابت مناسب است (منظور از کار انعطاف‌پذیر، کاری است که احتمال تغییر در میانه مسیر زیاد باشد).
\item
 مدیریت فرآیندها ساده‌تر است.
\item
 نسبت به سایر مدل‌ها به برنامه‌ریزی کمتری نیاز دارد.
\item
 چون به اعضا و روابط بین آنها بیشتر از فرآیند‌ها و ابزارها توجه می‌شود، ارتباط موثرتری بین افراد شکل می‌گیرد.
\item
 نسخه‌های قابل اجرایی از نرم‌افزار در دوره‌های کوتاه با فاصله زمانی کم آماده می‌شوند. 
\end{itemize}

معایب:
\begin{itemize}
\item
به دلیل نبود مستند‌سازی، انتقال پروژه به اعضای جدید تیم دشوار خواهد بود.
\item
اتکای بالای این مدل به ارتباط میان افراد و ارتباط موثر با مشتری می‌تواند یک نقطه‌ضعف باشد (در صورت نبود درک متقابل ممکن است پروژه از مسیر خود خارج شود).
\item
در مواردی که نرم‌افزار در نظر گرفته شده برای انتهای دوره نسخه مهمی باشد، پیش‌بینی نیازمندی‌ها و برنامه‌ریزی فاز دشوار خواهد بود.  
\end{itemize}
 
\end{itemize}

\subsection*{مراجع}

\begin{latin}
	\begingroup
	\renewcommand{\section}[2]{}%
	
\begin{thebibliography}{9}
%   https://www.student.unsw.edu.au/how-do-i-cite-electronic-sources

	\bibitem{2-1}
	Gregory, et al. “What Are the Software Development Models.” Try QA, \url{http://tryqa.com/what-are-the-software-development-models} 

	\bibitem{2-2}
	Shiklo, Boris. “8 Software Development Models: Sliced, Diced and Organized in Charts.” \textit{ScienceSoft Footer Icon}, ScienceSoft, 7 Nov. 2022, \url{https://www.scnsoft.com/blog/software-development-models\#Kanban}

	\bibitem{2-3}
	“7 Software Development Models Comparison: How to Choose the Right One?” \textit{Inoxoft}, 26 Oct. 2022, \url{https://inoxoft.com/blog/7-software-development-models-comparison-how-to-choose-the-right-one/\#6-agile-model}

	
\end{thebibliography}
\endgroup
\end{latin}

} 