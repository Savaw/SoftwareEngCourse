%نام و نام خانوادگی:
%شماره دانشجویی: 
\مسئله{}
یکی از جنبه‌های مورد بحث در متدولوژی
\lr{XP}،
 \lr{Pair programming} 
 یا برنامه‌نویسی دو نفره است. منظور از این مفهوم چیست و چه مزیتی دارد؟

\پاسخ{

منظور از برنامه‌نویسی دو نفره این است که دو نفر از اعضای تیم پشت یک کامپیوتر قرار گرفته و با یک‌دیگر روی یک مسئله هم‌کاری کنند. به طور معمول یکی از افراد اختیار موس و کیبرد را دارد و مشغول نوشتن کد و درگیر جزئیات است و فرد دیگر از دید کمی بالاتر به طور نقادانه به مسئله نگاه کرده و به استراتژی‌های مناسب برای پیش‌ رفتن فکر می‌کند. هم‌چنین چون از دید کلی‌تری نگاه می‌کند، مشکلاتی که وجود دارد و خطاهایی که رخ می‌دهد را راحت‌تر می‌بیند و مطرح می‌کند. لازم است به طور منظم جای نقش دو فرد جابه‌جا شود و هر دو نفر هر دو فعالیت را انجام دهند. هم‌چنین هر دو فرد به طور هم‌زمان به چالش‌های پیش آمده فکر می‌کنند و از طوفان فکری برای پیدا کردن راه‌حل استفاده می‌کنند.

از جمله مزیت‌های برنامه‌نویسی دو نفره این است که بسیاری از خطاهای برنامه در همان زمان نوشتن برنامه شناسایی می‌شوند. این مسئله کمک می‌کند از رفت و برگشت کد بین تیم توسعه‌دهنده و تیم تضمین کیفیت (\lr{QA}) جلوگیری شود که باعث افزایش بهره‌وری تیم می‌شود. هم‌چنین احتمال این که محصول خطایی در زمان اجرا پیدا کند کم‌تر می‌شود که باعث بهبود کیفیت محصول و بالا رفتن پایداری محصول شده و از هزینه‌های ناخواسته جلوگیری می‌کند.

مزیت دیگر این روش این است که چون دو نفر روی یک مسئله فکر می‌کنند و با استفاده از طوفان فکری و بحث و گفت‌وگو با یک‌دیگر به یک راه حل می‌رسند، ایده‌های بهتری برای حل مسئله نسبت به حالتی که یک نفر روی آن کار کند پیدا می‌شود. هم‌چنین معمولا زمان کم‌تری نیز برای رسیدن به ایده‌ی بهینه صرف می‌شود.

از طرف دیگر، وقتی دو نفر با یک دیگر برنامه‌نویسی می‌کنند، کدها تمیزتر، خواناتر و یک‌دست‌تر می‌شود. هم‌چنین افراد بییشتری از تیم روی محصولات تیم تسلط خواهند داشت. این ویژگی‌ها کمک می‌کند نگه‌داری محصولات در تیم راحت‌تر و بهتر انجام شود.

مزیت دیگر این است که از آن‌جایی که هر کدام از این دو نفر تجربه‌ی متفاوتی در مهندسی نرم‌افزار و هم‌چنین کار با محصولات تیم دارند، وقتی با یک‌دیگر کار می‌کنند این تجربه‌ها به اشتراک گذاشته می‌شود و در نتیجه اعضای تیم دانش فنی بیشتری هم درباره‌ی محصولات تیم و هم به طور کل در زمینه‌ی مهندسی نرم‌افزار پیدا خواهند کرد. یادگیری بیشتر هم باعث کیفیت بالاتر محصولات شده و هم باعث بالا رفتن رضایت اعضای تیم می‌شود.

یکی از ویژگی‌های خوب دیگر برنامه‌نویسی دو نفره این است که اعضای تیم لذت بیشتری از کارشان می‌برند و کم‌تر خسته می‌شوند که باعث بالا رفتن بهره‌وری و رضایت در تیم می‌شود.

از طرف دیگر صحبت اعضا با یک‌دیگر هنگام انجام برنامه‌نویسی دو نفره باعث می‌شود افراد تیم به هم نزدیک‌تر شوند که این باعث بهبود ارتباطات بین اعضای تیم می‌شود که از نیازهای اصلی یک تیم توسعه محصول است. هم‌چنین اطلاعات در تیم بهتر پخش می‌شود که این باعث هماهنگی بیشتر و یک‌دست‌تر شدن تیم می‌شود.


در نهایت، مزیت‌های برنامه‌نویسی دو نفره را می‌توان به صورت زیر خلاصه کرد:

\begin{itemize}
	
	\item
	پیدا کردن راه‌حل‌های بهتر برای مسائل
	
	\item
	بالا رفتن کیفیت کد
	
	\item
	بالا رفتن کیفیت نهایی محصول و کاهش نرخ خطا
	
	\item
	آشنایی بیشتر هر فرد با هر کدام از محصولات تیم 
	
	\item
	تسلط تعداد بیشتری از افراد روی هر قسمت از محصولات تیم
	
	
	\item 
بالا رفتن بهره‌وری تیم
	
	\item
	افزایش اشتراک دانش در تیم و بالا رفتن میزان یادگیری اعضای تیم
	
	\item
	بالا رفتن رضایت اعضا
	
	\item 
	بهبود ارتباطات درون تیمی
	
\end{itemize}


\subsection*{مراجع}

\begin{latin}
	\begingroup
	\renewcommand{\section}[2]{}%
	
	\begin{thebibliography}{9}
		%   https://www.student.unsw.edu.au/how-do-i-cite-electronic-sources
		
				\bibitem{5-3}
		Cockburn, A., \& Williams, L. (2000). The costs and benefits of pair programming. Extreme programming examined, 8, 223-247.
		
		
		\bibitem{5-1}
		Don Wells,
		\textit{Extreme Programming},
		Pair Programming,
		accessed 9 November 2022,
		\url{http://www.extremeprogramming.org/rules/pair.html}
		

		
		\bibitem{5-2}
		\textit{tutorials point},
		Extreme Programming - Pair Programming,
		accessed 9 November 2022,
		\url{https://www.tutorialspoint.com/extreme\_programming/extreme\_programming\_pair\_programming.htm}
		

	\end{thebibliography}
	\endgroup
\end{latin}

}